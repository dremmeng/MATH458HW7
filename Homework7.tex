\documentclass{article}

\usepackage[margin=1in]{geometry}
\usepackage{fancyhdr, lastpage}
\usepackage{tikz}
\usepackage{amsmath,amssymb,amsthm}
\usetikzlibrary{calc}
\usepackage{enumitem}
\usepackage{soul}

% Universes
\newcommand{\NN}{\mathbb{N}}
\newcommand{\ZZ}{\mathbb{Z}}
\newcommand{\QQ}{\mathbb{Q}}
\newcommand{\RR}{\mathbb{R}}
\newcommand{\CC}{\mathbb{C}}

% Groups commands
\newcommand{\inv}{^{-1}}
\newcommand{\lcm}{\mathrm{lcm}}
\newcommand{\lr}[1]{\langle #1 \rangle}
\newcommand{\Inn}{\mathrm{Inn}}
\newcommand{\iso}{\cong}
\newcommand{\normal}{\triangleleft}

%%%%%%%%%%%%%%%%%%%%%%%%%%%%%%%%%%%%%%%%%%%%%%%%%%%%%%%%%%%%%%
\setlength{\parindent}{0cm}
\pagestyle{fancy}
\lhead{MATH458 Abstract Algebra}
\rhead{Homework 7}

%%%%%%%%%%%%%%%%%%%%%%%%%%%%%%%%%%%%%%%%%%%%%%%%%%%%%%%%%%%%%%
\begin{document}
\section*{Homework 7}

Unless otherwise indicated, you must justify all answers/steps. See the Canvas assignment for more information about the homework requirements. 

\begin{enumerate}

    \item Let $N$ be a normal subgroup of a group $G$. Use properties of group homomorphisms to show that every subgroup of $G/N$ has the form $H/N$, where $H$ is a subgroup of $G$. 
    
    \item The Second Isomorphism Theorem states:
        \begin{quote}
            If $K$ is a subgroup of $G$ and $N$ is a normal subgroup of $G$, then $K/(K\cap N)\iso KN/N$.
        \end{quote}

        \begin{enumerate}
            \item Draw a diagram (similar to a subgroup lattice) that shows the relationship between the groups $G$, $K$, $N$, $KN$, $K\cap N$, and $\{e\}$. 

            \item Implied in the conclusion of the theorm is that $KN\le G$, $K\cap N\normal K$ and $N\normal KN$. Justify why this is true given the hypotheses.
    
            \item Prove the theorem.

        \end{enumerate}
        
    \item The set $G=\{1,4,11,14,16,19,26,29,31,34,41,44\}$ is  group under multiplication modulo 45. 
        \begin{enumerate}
            \item Write $G$ as an internal direct product of cyclic subgroups of prime-power order. (No justification needed here.)
            
            \item Determine the isomorphism class of $G$ by writing $G$ as an external direct product of cyclic groups of prime-power order. (No justification needed here.)
            
            \item Justify your answers to parts (a) and (b).
            
        \end{enumerate}
    
    \item Let $G$ be an abelian group of order $p^n$ for some prime $p$ and positive integer $n$. Prove that $G$ is cyclic if and only if $G$ has exactly $p-1$ elements of order $p$. 

    \end{enumerate}
\end{document}